\documentclass[10pt]{article}

\title{Data Science 210 Final Project}
\author{Thomas Kwashnak}
\date{Fall 2021}
\usepackage{amsmath}
\usepackage{soul}
\usepackage{algorithm2e}
% \usepackage[citestyle=alphabetic,bibstyle=authortitle]{biblatex}
\usepackage{listings}
\usepackage{enumitem}
\usepackage[margin=1in]{geometry}


\begin{document}
\maketitle
\setlength{\parindent}{0pt}.
\setlength{\parskip}{\baselineskip}
\lstset{numbers=left, numberstyle=\footnotesize, frame=l} 
\RestyleAlgo{ruled}


% These sections are good for a start, however change them up as needed
\section{Background}
% State Algorithm chosen and state what task we'll compete
% Brief outline on mathmatics behind algorithm
% Summary of Dataset
\section{Design}
During this section, a variety of variables and variable syntax will be used. The following is a list that explains each variable. Remember that for a given "layer", the output is considered the 0th layer, the hidden layer is considered the 1st layer, and the input is considered the 2nd layer.
% \begin{center}
%     \begin{tabular}{|l|l|}
%         \hline
%         $\vec{V}$ & The input values of the neural Network\\
%         $\vec{v_{i}}$ & The $i$th input value of the Input vector $\vec{V}$\\
%         $W^i$ & The weight matrix for a given layer $i$ on inputs from layer $i+1$. \\
%         \hline
%     \end{tabular}
% \end{center}
\begin{description}[style=nextline]
    \item[$\vec{y}^n$] The output values of the nodes in the $n^{\text{th}}$ hidden layer. 
    \item[$y^n_i$] The output value of node $i$ in the $n^{\text{th}}$ hidden layer.
    \item[$y$] By default, if y is left alone, it represents the output layer, known as $y^0_0$ 
    \item[$\hat{y}$] The expected final output 
    \item[$W^n$] The weight matrix for a given layer $n$, as it applies to the values $\vec{y}^{n+1}$ (the outputs of the previous layer) 
    \item[$w^n_{i,j}$] The weight value for the value passed from node $j$ of layer $n+1$ as it is passed to node $i$ of layer $n$
\end{description}
\subsection{Derivations and Equations}

\subsubsection{The $\sigma$ function}
The sigmoid activation function that will be used for all nodes (apart from the input layer, which uses a linear activation function) can be explained as follows:
$$\sigma(n) = \frac{1}{1 + e^{-n}}$$
$$\sigma'(n) = \sigma(n) * (1 - \sigma(n))$$
When using this function on a matrix or vector, we simply just apply the sigma function for all individual values in the matrix/vector

\subsubsection{Forward Feeding Equation}
This equation represents what the output will be for a given input $\vec{y^2}$. The basic principal is derived from the following relationship for the ouputs of layer $n$:
$$\vec{y^n} = \sigma(W_{n} * \vec{y^{n+1}})$$
We can chain this rule to find the output vector. Since the output vector is a 1x1 matrix, it will result in a scalar value
$$y = \sigma(W_0 * \sigma(W_1 * \vec{y^2}))$$
We can then simplify the first layer by writing the manual multiplication:
$$y = \sigma(\sum_i{(w^0_{0,i} * \sigma(W^1 * \vec{y^2})_{i,0})})$$
\subsubsection{The Loss Function}
The loss function is the overall function we wish to minimize, and is as follows:
$$L = (y^0_0 - \hat{y})^2$$

\subsubsection{Layer 1 Derivative}
In back propagation, we need to find the derivatives for each of the weights in the first layer. The derivative can be simplified using the chain rule as follows:
$$\frac{\delta L}{\delta w^0_{0,i}} = \frac{\delta L}{\delta y} * \frac{\delta y}{\delta w^0_{0,i}}$$
We can factor this out to be the following. Note that $\sigma(\ldots) = \sigma(\sum_i{(w^0_{0,i} * \sigma(W^1 * \vec{y^2})_{i,0})}) = y$.
$$\frac{\delta L}{\delta w^0_{0,i}} = 2(y - \hat{y}) * (\sigma'(\ldots)) * (\sigma((W^1 * \vec{y^2})_{i,0}))$$
$$\frac{\delta L}{\delta w^0_{0,i}} = 2(y - \hat{y}) * (\sigma(\ldots)(1 - \sigma(\ldots))) * (\sigma(\sum_h{(w^1_{i,h}* y^2_{h})})$$
$$\frac{\delta L}{\delta w^0_{0,i}} = 2(y - \hat{y}) * (y * (1 - y)) * (\sigma(\sum_h{(w^1_{i,h} y^2_{h})})$$
$$\frac{\delta L}{\delta w^0_{0,i}} = 2(y - \hat{y}) * (y - y^2) *  (\sigma(\sum_h{(w^1_{i,h} *y^2_{h})})$$
And thus, we have the derivative of a given weight in relation to the network.

\subsubsection{Layer 2 Derivative}
The last Layer 2 derivative is based on the derivative of the 1st layer's derivative.
$$\frac{\delta L}{\delta w^1_{i,j}} = \frac{\delta L}{\delta w^0_{0,i}} * \frac{\delta w^0_{0,i}}{\delta y^1_i} * \frac{\delta y^1_i}{\delta w^1_{i,j}}$$
Thus, with using $\frac{\delta L}{\delta w^0_{0,i}}$, we can derive $\frac{\delta L}{\delta w^1_{i,j}}$. We will keep $\frac{\delta L}{\delta w^0_{0,i}}$ in as itself, since in our algorithms we will be able to store this value from the previous step. In this instance, $\sigma(\ldots) = \sigma(W^1 * \vec{y^2}) = y^1_i$.
$$\frac{\delta L}{\delta w^1_{i,j}} = \frac{\delta L}{\delta w^0_{0,i}} * (2(y-\hat{y})(y - y^2)(\sigma'(\ldots))) * (w^1_{i,j} * y^2_{j})$$
$$\frac{\delta L}{\delta w^1_{i,j}} = \frac{\delta L}{\delta w^0_{0,i} }* (2(y - \hat{y})(y - y^2)(\sigma(\ldots)(1 - \sigma(\ldots)))) * (w^1_{i,j} * y^2_{j})$$
$$\frac{\delta L}{\delta w^1_{i,j}} = \frac{\delta L}{\delta w^0_{0,i}} * (2(y - \hat{y})(y - y^2)(y^1_i (1 - y^1_i))) * (w^1_{i,j} * y^2_{j})$$
$$\frac{\delta L}{\delta w^1_{i,j}} = \frac{\delta L}{\delta w^0_{0,i}} * (2(y - \hat{y})(y - y^2)((y^1_i) - (y^1_i)^2) * (w^1_{i,j} * y^2_{j})$$


\subsection{Data Importing}
Since the data was given in a string csv format, an additional pytohn script was created to import the data into two matricies, one contianing the inputs and one contianing the expected outputs.


% Pseudocode, referenced previous algorithms (just include their pseudocode)
\section{Implementation}
% Reference methods / describe code?
\section{Validation}
\section{Reflection}

\section{Notes}
Forward Feeding Algorithm
$$S(W_0 * f(W_i * \vec{\text{In}})) = \text{Out}$$

\section{Equations}
$$s(x) = \frac{1}{1 + e^{-x}}$$


% $$\frac{\delta L}{\delta w_{i,j}} = 2(O_i - y_i)(O_i)(1 - O_i)(y_j)$$

\end{document}